%-------------------
%Beginn des Kopfbereiches
%-------------------

%Wir verwenden eine DIN-A4-Seite und die Schriftgr��e 12.
\documentclass[a4paper,
               12pt,
               listsleft,
               %openany,               % Nach neuem Kapitel eine neue Seite beginnen
               oneside,
               openbib,
               smallheadings]{scrartcl}

%Diese drei Pakete ben�tigen wir f�r die Umlaute, deutsche (oder franz�sische) Silbentrennung etc.
%Apple-Nutzer sollten anstelle von \usepackage[latin1]{inputenc} das Paket \usepackage[applemac]{inputenc} verwenden
\usepackage[T1]{fontenc}
\usepackage[latin1,ansinew]{inputenc}
%serifenlose Schrift
\usepackage{helvet} %Helvetica clone
\renewcommand{\familydefault}{\sfdefault} 

\usepackage{ngerman}
%\usepackage[french]{babel}

%f�r die Nutzung f�r Geisteswissenschaften, etc.
%\usepackage{jurabib}

%Das Paket wird f�r die anderthalb-zeiligen Zeilenabstand ben�tigt
\usepackage{setspace}
\usepackage[marginal]{footmisc}

%Einr�ckung eines neuen Absatzes
\setlength{\parindent}{0em}

%Definition der R�nder (Texabmessungen oben,unten,links,rechts in mm ohne Kopfzeile)
\usepackage[paper=a4paper,
            includehead,
            left=40mm,
            right=20mm,
            top=20mm,
            bottom=20mm]
            {geometry} 

% linksb�ndige Fu�boten
\deffootnote{1.5em}{1em}{%
\makebox[1.5em][l]{\thefootnotemark}}

%Abstand der Fu�noten
\deffootnote{1em}{1em}{\textsuperscript{\thefootnotemark\ }}

%Serifenlose Fu�noten
\setkomafont{footnote}{\sffamily}
\hyphenation{Tra-jektorien}

% Durchg�ngig nummerierte Fu�noten
%\usepackage[flushmargin,multiple]{footmisc}
\usepackage{url}
\usepackage{color}
\usepackage[automark,
            plainheadsepline,                           % Markieren,Kopfzeile auch bei Chapter
          nouppercase]
            {scrpage2} 

\makeatletter
\usepackage{remreset}
\@removefromreset{footnote}{chapter}
\makeatother

%Regeln, bis zu welcher Tiefe (section,subsection,subsubsection) �berschriften angezeigt werden sollen (Anzeige der �berschriften im Verzeichnis / Anzeige der Nummerierung)
\setcounter{tocdepth}{3}
\setcounter{secnumdepth}{3}

%Einr�ckung eines neuen Absatzes
\setlength{\parindent}{0em}

%\renewcommand*{\chapterheadstartvskip}{\vspace*{-1\baselineskip}}            % Chapter nach oben r�cken
\usepackage{typearea}
\clubpenalty = 10000                                    % Schusterjungen vermeiden
\widowpenalty = 10000                                   % Hurenkinder vermeiden

%gepunktete Linien im Inhaltsverzeichnis
\usepackage[titles]{tocloft}
\renewcommand{\cftsecleader}{\cftdotfill{\cftdotsep}}

% Einstellungen f�r PDF Dokumente Links,Farben und Co
\usepackage{color}
\definecolor{darkblue}{rgb}{0,0,.5}
\usepackage[final,
%           dvips,                                      % auskommentieren, falls pdflatex verwendet wird
            pdftitle=pdftitle,		             
            pdfauthor=author_name,                       
            hypertexnames=true,
            breaklinks=true,
            colorlinks=true,
            linkcolor=darkblue,
            menucolor=darkblue,
            pagecolor=darkblue,
            urlcolor=darkblue,
            citecolor=darkblue,
            plainpages=false,                           % soll doppelte Seitenbezeichnung verhindern,
            pdfpagelabels,                              % funktioniert bei Titelblatt nicht
            bookmarks=true,
%           bookmarksopen=true,                         % Lesezeichen komplett offen
%           bookmarksopenlevel={1},
            bookmarksnumbered=true,                     % Lesezeichen werden mit Kapitelnummern angezeigt
            ]{hyperref}

%Umbrechen von Internetadressen in Fu�noten
\usepackage[hyphenbreaks]{breakurl}
\usepackage{graphicx}                                   % Grafikpaket zum Einfuegen von Bildern
\usepackage{subfig}                                     % Grafikpaket f�r Nebeneinanderstehende Bilder
\renewcommand{\figurename}{Abb.}                        % umbenennen der Abbildungsbeschriftung in Abb.
%-------------------
%Ende des Kopfbereiches
%------------------- 
\usepackage{listings}
\usepackage{color}

\definecolor{dkgreen}{rgb}{0,0.6,0}
\definecolor{gray}{rgb}{0.5,0.5,0.5}
\definecolor{mauve}{rgb}{0.58,0,0.82}

\lstset{frame=tb,
  language=Java,
  aboveskip=3mm,
  belowskip=3mm,
  showstringspaces=false,
  columns=flexible,
  basicstyle={\small\ttfamily},
  numbers=none,
  numberstyle=\tiny\color{gray},
  keywordstyle=\color{blue},
  commentstyle=\color{dkgreen},
  stringstyle=\color{mauve},
  breaklines=true,
  breakatwhitespace=true,
  tabsize=3
}
% Taken from Lena Herrmann at 
% http://lenaherrmann.net/2010/05/20/javascript-syntax-highlighting-in-the-latex-listings-package
\usepackage{listings}
\usepackage{color}
\definecolor{lightgray}{rgb}{.9,.9,.9}
\definecolor{darkgray}{rgb}{.4,.4,.4}
\definecolor{purple}{rgb}{0.65, 0.12, 0.82}

\lstdefinelanguage{JavaScript}{
  keywords={typeof, new, true, false, catch, function, return, null, catch, switch, var, if, in, while, do, else, case, break},
  keywordstyle=\color{blue}\bfseries,
  ndkeywords={class, export, boolean, throw, implements, import, this},
  ndkeywordstyle=\color{darkgray}\bfseries,
  identifierstyle=\color{black},
  sensitive=false,
  comment=[l]{//},
  morecomment=[s]{/*}{*/},
  commentstyle=\color{purple}\ttfamily,
  stringstyle=\color{red}\ttfamily,
  morestring=[b]',
  morestring=[b]"
}

\lstset{
   language=JavaScript,
   backgroundcolor=\color{lightgray},
   extendedchars=true,
   basicstyle=\footnotesize\ttfamily,
   showstringspaces=false,
   showspaces=false,
   numbers=left,
   numberstyle=\footnotesize,
   numbersep=9pt,
   tabsize=2,
   breaklines=true,
   showtabs=false,
   captionpos=b
}