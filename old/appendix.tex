\appendix
\clearscrheadfoot
\clearpage
\thispagestyle{plain}
\pagenumbering{roman}
\section{Quellen \& Literatur}
%\bibliographystyle{jurabib}
%\bibliography{HA_Eigene_Vorlage}

\begin{list}{}{
\setlength{\leftmargin}{1cm}
\setlength{\itemindent}{-1cm}
}
\item Berger, Stefan: Narrating the Nation. Die Macht der Vergangenheit, in: APuZ (2008) Bd. 1-2, S. 7-13.
\item Brodersen, Kai/ Zimmermann, Bernhard (Hrsg.): Metzler. Lexikon der Antike, Stuttgart/ Weimar 2000.
\item Kert�sz, Imre: Europas bedr�ckende Erbschaft, in: APuZ (2008), Bd.1-2, S. 3-6.
\item Nelhiebel, Kurt: Der braune Faden. Vom deutschen Hass auf Juden und Kommunisten, in: Bl�tter f�r deutsche und internationale Politik (2010), H.4, S. 107-115.
\item Niethammer, Lutz: Kollektive Identit�t. Heimliche Quellen einer unheimlichen Konjunktur, Hamburg 2000.
\item Ostergard, Uffe: Der Holocaust und europ�ische Werte, in: APuZ (2008), Bd.1-2, S.25-30.
\item Schmahle, Wolfgang: Geschichte der europ�ischen Identit�t, in: APuZ (2008), Bd.1-2, S. 14-19.
\item Winkler, Heinrich August: Was hei�t westliche Wertegemeinschaft?, in: Internationale Politik (2007) Jg. 62, H. 4, S. 66-85.
\end{list}
\clearpage

\section{Internetquellen}
\thispagestyle{plain}
\begin{list}{}{
\setlength{\leftmargin}{1cm}
\setlength{\itemindent}{-1cm}
}
\item Antisemitismus in Deutschland 2009 - eine Chronik, \url{http://arug.de/component/option,com_docman/task,doc_view/gid,27/}, am 17.04.10.
\item Anti-Semitism stirs as Hungary goes to polls, in: The Times-Online vom 11.04.10, \url{http://www.timesonline.co.uk/tol/news/world/europe/article7094323.ece}, am 15.04.10.
\item Diederichsen, Diedrich: Kulturindustrie und Amerikakritik. Ist Adornos These, in der Kulturindustrie sei Kunst zur Massenkultur geworden, antiamerikanisch? Nein, aber die Frage ist trotzdem gut, in: Jungle World Nr. 38, vom 10.09.03, \url{http://jungle-world.com/artikel/2003/37/11345.html}, am 16.04.10.
\item European Navigator, \url{http://www.ena.lu/}, am 05.04.10.
\item Hauck, Barbara: Das Gleiche, aber anders. �ber die Konstituierung einer neuen europ�ischen Identit�t, in: Phase 2 (2004), H.11, \url{http://phase2.nadir.org/}, am 16.04.10.
\item Hennig, Eike: >>Einen Schlussstrich unter die nationalsozialistische
Vergangenheit ziehen<< Zur politischen Soziologie eines historischen Deutungsmusters, in: Einsicht 02. Bulletin des Fritz Bauer Instituts, \url{http://www.fritz-bauer-institut.de/publikationen/einsicht/Einsicht-02_hennig.pdf}, am 17.04.10.
\item \glqq Ich habe gelernt: Nie wieder Auschwitz\grqq, in Sueddeutsche-Online vom 24.01.05, \url{http://www.sueddeutsche.de/politik/612/351445/text/}, am 07.04.10.
\item Markovits, Andrei S.: Zwillingsbr�der. �ber europ�ischen Antisemitismus und Antiamerikanismus, in: Jungle World Nr. 48 vom 17.09.04 \url{http://jungle-world.com/artikel/2004/47/14110.html}, am 17.04.10.
\item Rede von Bundeskanzlerin Dr. Angela Merkel vor der Knesset am 18. M�rz 2008 in Jerusalem, in: Bulletin 26-1,  \url{http://www.bundesregierung.de/nn_774/Content/DE/Bulletin/2008/03/26-1-bk-knesset.html}\\, am 09.04.10.
\item Regisseur Lanzmann \glqq schockiert\grqq~ �ber Krawalle bei Israel-Film, in: Spiegel-Online, vom 19.11.09, \url{http://www.spiegel.de/kultur/gesellschaft/0,1518,661980,00.html}, am 17.04.10.
\item Wir sind alle Amerikaner, in: Die Welt-Online vom 13.09.01, \url{http://www.welt.de/print-welt/article475606/Wir_sind_alle_Amerikaner.html}, am 06.04.10.
\end{list}
\clearpage

\section{Erkl�rung}
\thispagestyle{plain}
Hiermit erkl�re ich, dass ich die vorliegende Arbeit selbst angefertigt und alle von mir benutzten Hilfsmittel und Quellen angegeben habe; alle w�rtlichen Zitate und Entlehnungen aus fremden Arbeiten sind als solche gekennzeichnet.\\
\\
\\
Ort, den Abgabedatum\\
\\
\\
\rule{60mm}{0,2mm} \\
\hspace{50mm}Name
\clearpage